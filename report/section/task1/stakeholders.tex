

\section{Xác định ngữ cảnh của dự án UWC 2.0}
    \subsection{Các bên liên quan (Relevant Stakeholders)}
        \begin{enumerate}
            \item Công ty cung cấp dịch vụ thu dọn rác Y, ở vai trò là người quản lý.
            \item Công ty cung cấp dịch vụ thu dọn rác Y, ở vị trí là người sở hữu phần mềm UWC 2.0.
            \item Tổ chức X, phát triển phần mềm UWC 2.0.
            \item Công nhân sử dụng phần mềm UWC 2.0.
            \item Các bên liên quan khác (chính phủ, người dân).
        \end{enumerate}

    \subsection{Những mong muốn của các bên liên quan}
        \begin{enumerate}
            \item Công ty cung cấp dịch vụ thu dọn rác Y, ở vai trò là người quản lý, họ mong muốn phần mềm UWC 2.0 sẽ mang lại cho công ty:
            \begin{itemize}
                \item[-] Khả năng quản lý nhân lực, kiểm tra một cách hiệu quả cũng như thường xuyên cập nhật thông tin về sức chứa của các điểm tập trung rác thải (MCPs) và những thông tin nhằm phục vụ nhu cầu bảo trì các phương tiện vận chuyển, trang thiết bị của công ty.
                \item[-] Nâng cao hiệu suất thu gom rác thải của công ty.
                \item[-] Hệ thống khi được đưa vào hoạt động có độ tin cậy cao, hoạt động tốt trong mọi tình huống.
            \end{itemize}

            \item Công ty cung cấp dịch vụ thu dọn rác Y, ở vị trí là người sở hữu phần mềm UWC 2.0, họ sẽ có những mong muốn:
            \begin{itemize}
                \item[-] Khả năng mở rộng phạm vi hoạt động của hệ thống, không chỉ là trong một vùng, một quận cố định, mà có thể xử lý trong phạm vi toàn thành phố.
                \item[-] Chi phí và lợi nhuận luôn là một trong những vấn đề ưu tiên hàng đầu cần phải được cân nhắc. Khi được đưa vào sử dụng, UWC 2.0 phải đảm bảo tối ưu hóa về mặt nhiên liệu để từ đó giảm thiểu được chi phí phải bỏ ra. Thêm vào đó là chi phí để bảo trì hệ thống cũng phải phù hợp với công ty.
                \item[-] Trước đây công ty đã có sẵn hệ thống UWC 1.0 và công ty mong muốn có thể tận dụng lại tối đa dữ liệu có sẵn từ phiên bản trước và khả năng tương thích ngược giữa phiên bản 1.0 và 2.0.
                \item[-] Đem lại những kết quả tích cực đến công tác xử lý và tái chế rác thải trong cộng đồng, tạo ra môi trường sống xanh, sạch, đẹp cho người dân.
            \end{itemize}

            \item Tổ chức X, phát triển phần mềm UWC 2.0 có mong muốn:
            \begin{itemize}
                \item[-] Xây dựng được hệ thống có thể làm được và làm tốt những yêu cầu tối thiểu được yêu cầu bởi công ty chủ quản, xa hơn nữa là hoàn thiện hệ thống một cách tối ưu.
                \item[-] Hệ thống khi đến tay khách hàng phải dễ dàng bảo dưỡng, để nếu trong tương lai nếu khách hàng có thuê một công ty khác sửa chữa, bảo dưỡng hệ thống cũng thuận tiện hơn.
            \end{itemize}

            \item Công nhân sử dụng phần mềm UWC 2.0 mong muốn:
            \begin{itemize}
                \item[-] Vì đa phần người dùng là các cô chú trung niên, lớn tuổi, phần mềm nên có giao diện thân thiện, dễ dàng tiếp cận và làm quen.
                \item[-] Có thể sử dụng phần mềm trên nhiều thiết bị khác nhau như máy tính, điện thoại hay máy tính bảng.
                \item[-] Phần mềm phải sử dụng ổn định, ít giật lag và độ tin cậy cao.
            \end{itemize}

            \item Các bên liên quan khác (chính phủ, người dân): mong muốn phần mềm sẽ mang lại những ảnh hưởng tích cực đến công tác quản lý và tái sử dụng rác thải trong phạm vi khu vực và trong cả nước.
        \end{enumerate}

    \subsection{Vấn đề các bên liên quan đang gặp phải}
        \begin{enumerate}
            \item Công ty cung cấp dịch vụ thu dọn rác Y hiện đang gặp phải các vấn đề:
            \begin{itemize}
                \item[-] Công ty đang thiếu khả năng kiểm tra tình trạng trang thiết bị và phương tiện nên việc hoạt động chưa hiệu quả.
                \\
                \emph{\underline{Ví dụ}}: Chưa kiểm soát được tải trọng, sức chứa, tiêu thụ nhiên liệu dẫn tới sự phân bố xe chưa hợp lý khi hoạt động.

                \item[-] Việc quản lý nhân lực của công ty chưa hiệu quả:
                \begin{itemize}
                    \item[+] Công ty chưa theo dõi được tiến độ làm việc, hiệu suất làm việc của nhân viên.
                    \item[+] Việc phân công công việc chưa hợp lý gây mất công bằng và giảm hiệu suất công việc.
                \end{itemize}
            \end{itemize}

            \item Công ty cung cấp dịch vụ thu dọn rác Y, ở vị trí là người sở hữu hệ thống đang gặp phải các vấn đề:
            \begin{itemize}
                \item[-] Khả năng duy trì và nâng cấp hệ thống. Việc duy trì hệ thống đang có chi phí cao, chưa hợp lý.
                \item[-] Khi gặp các sự cố bất ngờ như  tai nạn, hư hỏng thiết bị thì việc liên lạc để điều phối nhân viên chưa kịp thời .
                \item[-] Công ty gặp bất tiện trong việc theo dõi và phân công lịch trình: lịch trình giấy, phân công thủ công. Điều này khiến công ty mất nhiều thời gian, nhân sự và việc phân công này chưa được tối ưu.
            \end{itemize}

            \item Các bên liên quan khác (chính phủ, người dân): Việc cải thiện môi trường chưa được tối ưu hoá, chi phí cao nhưng chưa hiệu quả dẫn đến lãng phí.
        \end{enumerate}

    \subsection{Lợi ích mà các bên liên quan có thể đạt được khi sử dụng hệ thống UWC 2.0}
        \begin{enumerate}
            \item Nhà cung cấp dịch vụ thu dọn rác Y ở vai trò quản lý:
            \begin{itemize}
                \item[-] Nắm bắt được cụ thể tình trạng hiện tại của các trang thiết bị và cơ sở vật chất, từ đó có thể đưa ra được những quyết định như nâng cấp, sửa chữa hay bổ sung trang thiết bị cho công nhân một cách kịp thời và hiệu quả.
                \item[-] Nhanh chóng theo dõi được tiến độ, hiệu suất làm việc của nhân viên. Đưa ra được phương án giải quyết sự cố kịp thời và công bằng.
            \end{itemize}

            \item Nhà cung cấp dịch vụ Y dưới góc độ chủ sở hữu hệ thống:
            \begin{itemize}
                \item[-] Tiết kiệm được chi phí duy trì và phát triển hệ thống.
                \item[-] Mang lại danh tiếng cho công ty trong mảng thu dọn rác thải.
            \end{itemize}

            \item Tổ chức X qua vai trò nhà phát triển phần mềm:
            \begin{itemize}
                \item[-] Tăng thu nhập cho công ty cũng như thu nhập của các cá nhân tham gia phát triển hệ thống.
                \item[-] Tăng kỹ năng của từng cá nhân.
                \item[-] Nâng cao danh tiếng, sự tin cậy cho doanh nghiệp.
            \end{itemize}

            \item Nhân viên  sử dụng phần mềm:
            \begin{itemize}
                \item[-] Dễ dàng sử dụng và làm quen.
                \item[-] Công sức làm việc của bản thân mỗi người được đánh giá công bằng và rõ ràng.
                \item[-] Khả năng liên lạc nhanh chóng giữa công nhân và nhân viên giám sát nhằm xử lý được những tình huống bất ngờ có thể xảy ra trong quá trình làm việc.
            \end{itemize}

            \item Người bị ảnh hưởng khác (chính phủ, người dân): Bảo vệ môi trường.
        \end{enumerate}
    \newpage
