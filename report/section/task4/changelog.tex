\section{Changelog}
    \begin{tblr}{
        width=1\linewidth,
        hlines, 
        vlines,
        colspec={X[2]X[8]},
        columns = {valign = m, },
        column{1} = {halign = c},
        row{1} = {halign = c, valign = m, bg = lightgray, fg = black},
        }
        {\textbf{Ngày}} & \textbf{Nội dung} \\
        \textbf{7/11/2022} & \textbf{\textit{Thêm:}} \newline
                            - Phân tích sự lựa chọn thiết kế kiến trúc ở Mục 9\newline
                            - Kiến thức của các kiến trúc ở Mục 9\newline
                            \textbf{\textit{Thay đổi:}} \newline
                            - Mô tả Kiến trúc hệ thống tại mục 9 \newline
                            \textbf{\textit{Sửa:}} \newline
                            - Class Diagram ở tên mối quan hệ Controller – Backofficer ở Mục 8\newline
                            - Kiến trúc hệ thống ở Mục 9\newline
                            - Chính tả ở mục 6, 7.2, 10.2\\
        \textbf{5/11/2022} & \textbf{\textit{Thêm:}} không có. \newline
                            \textbf{\textit{Thay đổi:}} \newline
                            - Mô tả thiết kế kiến trúc để xây dựng hệ thống. \newline
                            \textbf{\textit{Sửa:}} \newline
                            - Cập nhật lại thiết kế kiến trúc.\newline
                            - Cập nhật lại bảng mô tả.\\
        \textbf{4/11/2022} & \textbf{\textit{Thêm:}}\newline
                            - Lý do dùng kiến trúc MCV.
                            \textbf{\textit{Thay đổi:}} \newline
                            - Class Diagram cho task assignment module. \newline
                            - Mô tả thiết kế kiến trúc để xây dựng hệ thống. \newline
                            \textbf{\textit{Sửa:}} \newline
                            - Thay đổi lại class diagram. \\
        \textbf{26/10/2022} & \textbf{\textit{Thêm:}}\newline
                            - Thêm class diagram của các class Route Planning, Task Controller, UI, Route. \newline
                            \textbf{\textit{Thay đổi:}} \newline
                            - Các class diagram của các class. \newline
                            \textbf{\textit{Sửa:}} \newline
                            - Chỉnh lại nội dung class diagram của các class MCP, Staff, Back Officer, Vehicle. \\
        \textbf{25/10/2022} & \textbf{\textit{Thêm:}}\newline
                            - Thêm Task Component diagram cùng với đặc tả cho 2 task chính Task Assignment và Planning Route. \newline
                            \textbf{\textit{Thay đổi:}} \newline
                            - Task Component diagrams. \newline
                            \textbf{\textit{Sửa:}} không có.\\
        \textbf{24/10/2022} & \textbf{\textit{Thêm:}}\newline
                            - Thêm thiết kế kiến trúc, giải thích nguyên nhân, thêm các module. \newline
                            \textbf{\textit{Thay đổi:}} \newline
                            - Mô tả thiết kế kiến trúc. \newline
                            \textbf{\textit{Sửa:}} không có. \\
    \end{tblr}

    \begin{tblr}{
        width=1\linewidth,
        hlines, 
        vlines,
        colspec={X[2]X[8]},
        columns = {valign = m, },
        column{1} = {halign = c},
        row{1} = {halign = c, valign = m, bg = lightgray, fg = black},
        }
        {\textbf{Ngày}} & \textbf{Nội dung} \\
        \textbf{22/10/2022} & \textbf{\textit{Thêm:}}\newline
                            - Thêm ngữ cảnh thực tế tại thành phố Thủ Đức. \newline
                            \textbf{\textit{Thay đổi:}} \newline
                            - Xác định ngữ cảnh. \newline
                            \textbf{\textit{Sửa:}} không có.\\
        \textbf{13/10/2022} & \textbf{\textit{Thêm:}}\newline
                            - Thêm giải pháp ý niệm, mô tả chi tiết các thao tác thực hiện. \newline
                            - Thêm sequence diagram mô tả giải pháp. \newline
                            \textbf{\textit{Thay đổi:}} \newline
                            - Giải pháp ý niệm cho task Route planning. \newline
                            \textbf{\textit{Sửa:}} không có.\\
        \textbf{23/9/2022} & \textbf{\textit{Thêm:}}\newline
                            - Thêm use-case diagram của chức năng Task Assignment. \newline
                            - Thêm use-case diagram của chức năng Manage Resources. \newline
                            - Thêm use-case diagram tổng quát của hệ thống. \newline
                            - Thêm các requirements (non-functional và functional). \newline
                            - Thêm ngữ cảnh và các stakeholders của bài toán. \newline
                            \textbf{\textit{Thay đổi:}} \newline
                            - Use-case diagram. \newline
                            \textbf{\textit{Sửa:}} không có.\\
    \end{tblr}