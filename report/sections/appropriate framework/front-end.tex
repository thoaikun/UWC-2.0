\subsection{React Native}
    \subsubsection{Giới thiệu}
        \begin{figure}[h]
            \centering
            \includegraphics[width=0.9\linewidth]{images/appropriate framework/react-native.png}
            \caption{React Native}
        \end{figure}
        React Native là một framework của JavaScript được tạo ra  nhằm giúp các nhà phát triển triển khai ứng dụng lên nền tảng điện thoại bao gồm iOS và Android. Được xây dựng dựa trên React - một thư viện JavaScript được tạo ra bởi Facebook nhằm tối ưu hóa việc thiết kế giao diện, nhưng thay vì tập trung vào nền tảng web, React Native lại hướng mình tới nền tảng thiết bị di động bao gồm có điện thoại và máy tính bảng.
    
    \subsubsection{Ưu điểm}
        \begin{enumerate}
            \item Tận dụng được code đã viết. Bạn không cần phải phát triển riêng biệt cho hai nền tảng là Android và iOS, thay vào đó hơn 90\% có thể được tái sử dụng giữa hai nền tảng, từ đó giúp giảm thời gian, chi phí phát triển ứng dụng.
            \item Khi sử dụng react native thì sẽ ít phải sử dụng native code hơn.
            \item Có tính ổn định và khả năng tối ưu cao.
            \item Tập trung vào việc phát triển UI, giúp tạo ra giao diện có hiệu suất cao và khả năng tương tác tốt.
            \item Cộng đồng hỗ trợ rộng lớn.
            \item Duy trì ít code hơn, ít bugs hơn.
        \end{enumerate}
    \subsubsection{Nhược điểm}
        \begin{enumerate}
            \item Vẫn còn thiếu nhiều component quan trọng.
            \item Không thể thực hiện công việc xây dựng các chương trình iOS trên hệ điều hành Windows.
            \item Không sử dụng được để tạo ra các game có đồ họa cao và luật chơi phức tạp.
            \item Hiệu năng kém hơn so với khi sử dụng native app.
            \item Vì sử dụng JavaScript nên chưa có sự bảo mật an toàn.
            \item Khả năng tùy biến trong một số module chưa thật sự tốt.
        \end{enumerate}
\newpage