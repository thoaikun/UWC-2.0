\subsection{Kotlin}
    \subsubsection{Giới thiệu}
        \begin{figure}[h]
            \centering
            \includegraphics[width=0.99\linewidth]{images/appropriate framework/kotin.jpg}
            \caption{Kotlin}
        \end{figure}
        Kotlin là một ngôn ngữ ngữ dụng kiểu tĩnh dành cho Java Virtual Machine đã chính thức phát hành phiên bản 1.0. Nó được tạo ra bởi JetBrains, Kotlin cũng giống như nhiều ngôn ngữ lập trình không phải Java khác, tức là cũng sẽ chạy trên JVM và sử dụng các công cụ và thư viện hiện có của Java. Và ngược lại Java cũng có thể sử dụng các item được xây dựng trong Kotlin.
    \subsubsection{Ưu điểm}
        \begin{enumerate}
            \item Kotlin biên dịch thành JVM bytecode hoặc JavaScript - Giống như Java, Bytecode cũng là format biên dịch cho Kotlin. Bytecode nghĩa là một khi đã biên dịch, các đoạn code sẽ chạy thông qua một máy ảo thay vì một bộ xử lý. Bằng cách này, code có thể chạy trên bất kỳ nền tảng nào, khi nó được biên dịch và chạy thông qua máy ảo. Khi Kotlin được chuyển đổi thành bytecode, nó có thể truyền được qua mạng và thực hiện bởi JVM
            \item Có thể thay thế cho Java: Một trong những thế mạnh lớn nhất của Kotlin như là một ứng viên để thay thế cho Java là khả năng tương tác rất tốt giữa Java và Kotlin—bạn có thể thậm chí có code Java và Kotlin tồn tại song song trong cùng dự án. Vì Kotlin là hoàn toàn tương thích với Java, nên cũng có thể sử dụng phần lớn các thư viện Java và các framework trong dự án Kotlin của bạn.
            \item Miễn phí: Kotlin là mã nguồn mở nên không tốn kém gì để có thể sử dụng
            \item Code ngắn gọn hơn và dễ học: Kotlin tập trung nhiều hơn vào việc cú pháp dễ hiểu, dễ đọc để review, chúng có thể hoàn thành bởi những thành viên team chưa quen với ngôn ngữ này.
        \end{enumerate}
    \subsubsection{Nhược điểm}
        \begin{enumerate}
            \item Tốc độ biên dịch: Khi chúng ta biên dịch mã Kotlin lần đầu tiên, thì phải mất nhiều thời gian hơn Java. Java biên dịch nhanh hơn 15-20% so với Kotlin.
            \item Cộng đồng hỗ trợ hạn chế: Mặc dù là ngôn ngữ có thể sử dụng được toàn bộ cũng như thư viện của Java nhưng theo nhiều lập trình viên thì phiên bản chính chủ vẫn tốt hơn.  
        \end{enumerate}
